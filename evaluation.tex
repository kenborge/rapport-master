%!TEX root = main.tex

\chapter{Evaluation}
In this chapter we conclude our work by looking at the goals defined in the
introduction, and evaluate the results.
Section \ref{sec:evalres} evaluates the results of the project.
Section \ref{sec:conclusion} contains the final conclusion for the project and this report. 
Section \ref{sec:futurework} contains our thoughts on future implementation and improvements of our solution.


\section{Evaluation of Results}
\label{sec:evalres}
Our motivation for this project was to create an agent, using LIDA, that could play StarCraft: Brood War, which we succeeded in. We also discovered some of the problems inherent in the use of this cognitive architecture for this domain.

Because of limitations in the rule-based system currently used for action selection, we had to simplify some parts of our system. E.g., the decision about which building to build next was massively simplified compared to what a full, cognitive agent would have. Instead of using proper reasoning, we estimated whether we would be able to use the current amount of income.

\section{Conclusion}
\label{sec:conclusion}
We succeded in achieving what we set out to do, and while the performance was not excellent, we think it is a very good starting point for any future work in this area.

\section{Future Work}
\label{sec:futurework}
We believe that there are several areas to be worked on. Implementing more feature detectors, and more behaviours, might be the easiest and obvious way to go forward. Obviously missing behaviour is for example grouping and managing the individual units better, expanding to extra bases for additional income, as well as more complex build orders, instead of the extremely simple timing attack implemented in our bot.

Replacing the rule-based action selection module with a more advanced approach is also something to look into. Version 1.2 of the LIDA framework replaces the default rule-based action selection with a behaviour network, based on Maes' behaviour network.\cite{maes1989right}

A more significant change would be to use the various memory modules available more actively.\cite{franklin2007lida}. This could help with for example letting the bot remember more about the state of the game, for example if it has already built the necessary building necessary for a certain unit, or remember build orders (learned from earlier games).

Learning could also be implemented at several other levels, not just as long-term memory.
