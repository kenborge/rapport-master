%!TEX root = main.tex

\section{State of Current StarCraft AI}
\label{sec:stateofai}

Classical games like chess or tic-tac-toe are usually solved by AIs using a
single approach and searching through a single tree of game states, though
usually by optimizing the search and tree in various ways.

In comparison most approaches to AIs playing real-time strategy games usually
have to use domain knowledge do further subdivide the problem of playing the
game, because of the fine-grained simulations involved, and the various levels
of abstraction that is needed to get a successful AI. And especially when
approaching the way humans think about a problem more complex architectures
are needed.

\subsection{Decomposition of Problem}
Michael Buro in his 2003 call for research \cite{buro2003real} identified six
important sub-problems in real-time strategy games that he said would be
interesting for AI research to focus on. The problem subdivision also dictates
how the architecture has to be structured in order to solve them. It is
therefore important to have a clear view of what the different main sub-problems
are and how they will affect each other.

\begin{description}
  \item [Resource management.] 
  \item [Decision making with uncertainty.]
  \item [Spatial and temporal reasoning.]
   \item [Collaboration.]
  \item [Opponent modeling.]
  \item [Adversarial real-time planning.]
\end{description}

A more general and simplified breakdown of the problem of playing StarCraft can
be found in Ben Weber's presentation from the AIIDE 2010 StarCraft AI
Competition:\cite{weber2010aiide}

\begin{description}
  \item [Managing economy]
  \item [Expanding the tech tree]
  \item [Producing units]
  \item [Attack opponent]
\end{description}

\subsection{Architectures of Current StarCraft Agents}
Most bots have a rather similar, modular architecture.