%!TEX root = main.tex

\chapter{Introduction}
In this chapter we introduce our project. Section \ref{sec:background} presents the background and our motivation for the project, a short introduction to Starcraft and the API we will use and describes what we will do in this project and our main goals. Section \ref{sec:goals} explains the goals we hope to accomplish during the project. Section \ref{sec:structure} introduces the structure of this report.

\section{Background and Motivation}
\label{sec:background}

\subsection{The Problem}
Real-time strategy(RTS) games have been a complex problem to solve for artificial intelligence(AI) research for several years now. Ever since Michal Burro published an article in 2003\cite{buro2003real}, where he encouraged researchers to focus more of their expertise on the RTS domain, a lot of new methods have been applied to these problems with varying degrees of success. Some have been very successful and are now able to beat top level human players in real-time matches.\cite{campbell2002deep} 

After Burro's paper several research platforms emerged, including OpenRTS\cite{buro2003orts} and Wargus\cite{wargus}. Wargus is an open source WarCraft 2 clone, which uses the Lua scripting language for its agent creation, while OpenRTS is a platform originally created for AI research. These platforms have some limitations, however, because they are mostly developed by hobbyists and not professional game development companies, they lack both polish and wide-spread usage outside of academic environments. They also lack a lot in the way of individual unit management, path finding and generally the possibilities for complex strategies and tactics, because of the very simple game mechanics. 

As an answer to this the Brood War Application programming interface(BWAPI) was created. It supplies a comprehensive API for interacting with popular RTS-game StarCraft Brood War. The API lets developers easily create agents that play the game and can be run in tournaments against each other as well as against real players. It has gained a lot of popularity during the last years.\cite{bwapi} There are several competitions held each year, where programmers pit their BWAPI-based agents against each other, both for money and fame.\cite{sscait} But these artificial agents still have some way to go before they can match human players.\cite{eisbotvsfong}

While just winning the game is an interesting and worthy goal, it is according to Arrabales et al much more engaging for human players to play with other humans than with artificial agents.\cite{arrabales2009gamechars} We therefore think that developing human-like intelligence for playing computer games is something that should have a greater focus then it has had so far. Cognitive architectures are based around trying to emulate how the human consciousness works and acts in certain situations. There has been done very little research so far into using cognitive agents for playing RTS games, but they have been utilized with great success in first person shooter games previously. %%TODO mabye find a source for this?	

\section{Name}
\label{sec:name}
The name {\em Jantu} is hindi, and is translated as ``Pertaining to the merely sentient part of a creature, as distinguished from the intellectual, rational, or spiritual part; as, the animal passions or appetites.''\cite{hindijantu}, which reflects what we are trying to achieve with our work in this project.

\section{Project goals}
\label{sec:goals}
First and foremost we set out to create a proof of concept agent that could play the game StarCraft: Brood War, built using a cognitive architecture.

Secondly we wanted to see what kind of challenges present themselves when one implements an agent like this, and where there is room for improvement, and what the future might look like.

\section{Report Structure}
\label{sec:structure}
This report is structured into four chapters:
\begin{itemize}
\item Chapter 1: \textbf{Introduction} \\
This chapter describes the motivation and goal of the project as well as who
contributed and the general structure of the report.
\item Chapter 2: \textbf{Theory} \\
Here we present the domain we're working in; the real-time computer strategy game StarCraft: Brood War. We also present the cognitive architecture we're using.
\item Chapter 3: \textbf{Implementation} \\
This chapter presents our implementation.
\item Chapter 4: \textbf{Results} \\
This chapter presents how our implementation performed.
\item Chapter 5: \textbf{Evaluation} \\
Here we summarize and evaluate the work presented in this report.

\end{itemize}