%!TEX root = main.tex

\chapter{Introduction}
In this chapter we introduce our project. Section \ref{sec:background} presents the background and our motivation for the project and describes what we will do in this project and our main goals. Section \ref{sec:name} contains an explanation of the the name for the agent. Section \ref{sec:structure} lays out the structure of this report.

\section{Background and Motivation}
\label{sec:background}

\subsection{The Problem}
Real-time strategy(RTS) games have been a complex problem to solve for artificial intelligence(AI) research for several years now. Ever since Michael Buro published an article in 2003\cite{buro2003real}, where he encouraged researchers to focus more of their expertise on the RTS domain, a lot of new methods have been applied to these problems with varying degrees of success. Some have been very successful and are now able to beat top level human players in real-time matches.\cite{campbell2002deep}

After Burro's paper several research platforms emerged, including OpenRTS\cite{buro2003orts} and Wargus\cite{wargus}. Wargus is an open source WarCraft 2 clone, which uses the Lua scripting language for its agent creation, while OpenRTS is a platform originally created for AI research. These platforms have some limitations, however, because they are mostly developed by hobbyists and researchers and not professional game development companies, they lack both polish and wide-spread usage outside of academic environments. They also lack a lot in the way of individual unit management, path finding and generally the possibilities for complex strategies and tactics, because of the very simple game mechanics.

As an answer to this the Brood War Application programming interface (BWAPI) was created. It supplies a comprehensive API for interacting with the popular RTS-game StarCraft: Brood War. The API lets developers easily create agents that play the game and can be run in tournaments against each other as well as against real players. It has gained a lot of popularity during the last years.\cite{bwapi} There are several competitions held each year, where programmers pit their BWAPI-based agents against each other, both for money and fame.\cite{sscait} But these artificial agents still have some way to go before they can match human players.\cite{eisbotvsfong}

While just winning the game is an interesting and worthy goal, it is according to Arrabales et al much more engaging for human players to play with other humans than with artificial agents.\cite{arrabales2009gamechars} We therefore think that developing human-like intelligence for playing computer games is something that should have a greater focus then it has had so far. Cognitive architectures are based around trying to emulate how the human consciousness works and acts in certain situations. There has been done very little research so far into using cognitive agents for playing RTS games, but they have been utilized with great success in first person shooter games previously.\cite{arrabales2009gamechars}

\section{The Project}
In this project we will implement a proof of concept cognitive agent that plays StarCraft: Brood War. We will be using a cognitive framework called LIDA in our implementation. The first step in the project will be to integrate the LIDA framework with StarCraft and BWAPI, so the LIDA cognitive framework can control the StarCraft process. We also need to create a bridge between the in-game time reference called frames, and the frameworks time references called ticks. After accomplishing this we should be able to control the execution of StarCraft from our framework, so we can pause, resume and step frame by frame and tick by tick for debugging purposes.

Next step will be to create the proof of concept implementation of a cognitive bot that can play the game, with decent results. This will allow us to investigate what kind of challenges present themselves when one implements an agent like this, and where there is room for improvement, and what the future might look like.

\section{The Name}
\label{sec:name}
The name {\em Jantu} is hindi, and is translated as ``Pertaining to the merely sentient part of a creature, as distinguished from the intellectual, rational, or spiritual part; as, the animal passions or appetites.''\cite{hindijantu}, which reflects what we are trying to achieve with our work in this project.

\section{Report Structure}
\label{sec:structure}
This report is structured into five main chapters:
\begin{itemize}
\item Chapter 1: \textbf{Introduction} \\
This chapter describes the background and motivation, as well as what we hope to accomplish during the project. It also lays out the general structure of the report.
\item Chapter 2: \textbf{Theory} \\
In this chapter we presents the relevant theory for understanding the tools and technologies used during the project. We presents some general information about StarCraft as a game, we also present some theory on the cognitive architectures in general, and more detailed about the framework we will be using.
\item Chapter 3: \textbf{Implementation} \\
In this chapter we present how we integrated the framework with StarCraft and what features we implemented for our agent.
\item Chapter 4: \textbf{Results} \\
In this chapter we present the results of running the agent, how well it performed and some insight into how the bot ended up interpreting and acting in the given environment.
\item Chapter 5: \textbf{Evaluation} \\
Here we summarize and evaluate results and the project. We discuss what the agent did good and why it had problems in the situations it did. We also outline how we think the agent can be improved in the future.

\end{itemize}
